\section{Background}
\label{sec:background}
Network on chip is an interconnect approach that helps different IPs and subsystems in a chip to communicate with each other in an efficient and scalable manner. 
In this approach each processing element (PE) is connected to a switch and multiple switches are interconnected to form a network.
A PE could be a processor core, a DSP core or an IP block.
The network infrastructure helps in routing data from one PE to another in the form of data packets. 
Based on how the switches are interconnected, there are different NoC topologies such as mesh, torus, tree, ring, star, BFT etc.
In a mesh topology every switch, except the ones on the edges, is connected to 4 other neighboring switches.
A torus topology is similar to mesh but cyclic in nature.
In a binary tree, switches are arranged in a hierarchy.
Each switch has a parent node and two child nodes.
Unlike mesh and torus where each switch has a corresponding PE, in a tree topology only the switches at the bottom most level are connected to PEs.

For interconnected networks, an important performance parameter is the bisection bandwidth.
It is defined as the minimum bandwidth between two equal partitions of the network.
For a mesh topology, it is $\sqrt{n}*B$, where n is the number of switches and B is the bandwidth of a single link.
For torus, it is twice that of mesh but for a binary tree, it is only B.
To address this issue, instead of using a single link between switches more links can be used between them as we go higher in the tree hierarchy.
Such topology is called a fat tree~\cite{Leiserson1985}. 
Although this will improve the bisection bandwidth, the switches in the upper hierarchy becomes more and more complex.
We analyze whether using asynchronous switches with same link width can provide similar performance of fat trees while keeping relatively simpler switches.