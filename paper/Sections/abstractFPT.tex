\begin{abstract}
Binary tree topology generally fails to attract interconnected network implementations due to its low bisection bandwidth. 
Fat trees are proposed to alleviate this issue by using increasingly thicker links to connect switches towards the root node. 
This scheme is very efficient for computer networks, which use generic switches for interconnecting compute nodes. 
In a network on chip (NoC) context, especially for field programmable gate arrays (FPGAs), fat trees require more complex switches as we move higher in the hierarchy which restricts the maximum clock frequency.
This can offset the higher bandwidth achieved through using fatter links. 
In this paper we analyze the implementation of an asynchronous binary tree (AsyncBTree) based NoC, which achieves better bandwidth by varying the clock frequency between the switches as we move higher in the hierarchy. 
This scheme enables using simpler switch architecture enabling higher maximum frequency of operation. 
The effect on bandwidth and resource requirement of this architecture is compared with traditional and fat tree implementations for different number of nodes and different network traffic patterns.
\end{abstract}