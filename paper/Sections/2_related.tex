\section{Background}
\label{sec_background}

Demand for high performance and low power consumption have pushed engineers to integrate many computing resources such as GPUs, DSPs, memories and various IP blocks into a single chip referred as system on a chip (SoC). 
As the number of resources increase, their interconnection poses several challenges. 
Previously most of the SoC applications used a shared-bus interconnect because of its low cost and simplicity.
But it has its own limitations such as non-scalability and increased wire delays. 
At a specific instance, only one master can control the bus and an arbitrator is needed to manage concurrent bus requests.
This deteriorates the overall system performance as more number of resources are interconnected. In order to overcome this, an on-chip packet-switched network called Network-on-Chip was proposed.

Network on chip is an interconnect approach that helps different IPs and subsystems in an SoC to communicate with each other in a scalable manner. 
In this approach each processing element (PE) is connected to a switch and multiple switches are interconnected to form a network.
A PE could be a processor core, a DSP core or an IP block.
The network infrastructure helps in routing data from one PE to another in the form of data packets. 
This architecture enables concurrent communication between multiple PEs by exploiting hardware parallelism.

%
%Based how switches are interconnected, there are different NoC topologies such as mesh, torus, star, ring, butterfly fat tree (BFT) etc. 
%Mesh topology is a 2D arrangement of switches where each switch is connected to its neighbouring  switches and its corresponding PE. 
%Switches along the edges have 2 or 3 connections as there are only 2 or 3 switches adjacent to it. 
%Torus topology is similar to mesh but it is cyclic in nature. 
%Here all the switches, including the ones along the edges, will have four connections as shown in Fig.~\ref{fig:torus}. 
%Top most switches are connected to the bottommost and rightmost switches to the leftmost. 
%Star topology has a central hub to which all the switches are connected. 
%Communication between PE's is done through the central hub. 
There have been previous efforts to develop open-source NoC architectures specifically targeting FPGAs.
CONNECT NoC generator is the most popular among them~\cite{papa_connect_fpga2012}.
CONNECT is inspired by the fact that FPGAs have a large routing infrastructure available when compared to memory and logic elements and tries to exploit it. 
It supports different NoC topologies and uses a single stage pipeline mechanism  to minimize hardware and latency. 
It has low operating frequency and is still quite resource intensive as seen in Section~\ref{sec_results}. 
Split-merge is another NoC infrastructure developed at University of Pennsylvania~\cite{Huan2012}.
It tries to overcome the limited clock performance of CONNECT at the expense of few more resources.
Both CONNECT and Split-merge use credit based traffic control and requires the application developer to have special logic to manage credits.

Presently available open-source architectures are capable of providing relatively high throughput but are quite resource intensive.
Split-merge source code is available in BlueSpec and is challenging to customize.
Both NoCs use custom PE interfaces making it difficult to interface off-the-shelf IP cores.
Researchers have claimed to develop lean NoCs targeting FPGAs but are not publicly available~\cite{hoplite_fpl2015}.
The main motivation for this work is to provide a lite-weight NoC implementation with standard communication interface and high clock performance.
We aim to provide an end-to-end solution where NoC developers can easily interface it with a host computer for data communication.
We call this platform as OpenNoC.